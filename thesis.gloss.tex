\newacronym{psa}{PSA}{Pressure Swing Adsorption}
\newacronym{tsa}{TSA}{Temperature Swing Adsorption}
\newacronym{vsa}{VSA}{Vaccuum Swing Adsorption}
\newacronym{ca}{CA}{Cellulose Acetate}
\newacronym{cb}{CB}{Cigarette Butt}
\newacronym{ucb}{UCB}{Used Cigarette Butt}
\newacronym{ucf}{UCF}{Used Cigarette Filter}
\newacronym{tga}{TGA}{Thermogravimetric Analysis}
\newacronym{p-xrd}{P-XRD}{Powder X-ray Diffraction}
\newacronym{xps}{XPS}{X-ray Photoelectron Spectroscopy}
\newacronym{icp-oes}{ICP-OES}{Inductively Coupled Plasma-Optical Emission Spectrometry}
\newacronym{sem}{SEM}{Scanning Electron Microscopy}
\newacronym{se}{SE}{secondary electrons}
\newacronym{sei}{SEI}{secondary electron images}
\newacronym{bse}{BSE}{backscatter electrons}
\newacronym{bed}{BED}{backscatter electron detection}
\newacronym{tem}{TEM}{Tunneling Electron Microscopy}
\newacronym{edx}{EDX}{Electron-Dispersive X-ray Analysis}
\newacronym{psd}{PSD}{Pore Size Distribution}
\newacronym{dft}{DFT}{Density Functional Theory}
\newacronym{gcmc}{GCMC}{Grand Canonical Monte Carlo}
\newacronym{nldft}{NLDFT}{Non-Local Density Functional Theory}
\newacronym{ds}{DS}{degree of substitution}
\newacronym{sd}{SD}{sawdust}
\newacronym{nc}{NC}{sodium carboxymethyl cellulose}
\newacronym{bet}{BET}{Brunauer, Emmet and Teller theory}
\newacronym{iupac}{IUPAC}{International Union of Pure and Applied Chemistry}
\newacronym{mof}{MOF}{Metal Organic Framework}
\newacronym{pypuc}{pyPUC}{python Porosity Uptake Correlator}
\newacronym{ztc}{ZTC}{zeolite templated carbon}
\newacronym{dac}{DAC}{Direct Air Capture}

\newglossaryentry{activating agent}
{
    name=activating agent,
    description={Reagent used to develop porosity in a material. Also known as a \gls{porogen}}
    }
\newglossaryentry{porogen}
{
    name=porogen,
    description={See \gls{activating agent}}
    }
\newglossaryentry{porogenesis}{
    name=porogenesis,
    description={See \gls{activation}}
}
\newglossaryentry{htc}
{
    name=hydrothermal carbonisation,
    description={Carbonisation in water in a sealed container, typically of biomass}
    }
\newglossaryentry{hydrochar}
{
    name=hydrochar,
    description={The solid, carbonaceous product of \gls{htc}, composed of microspheres having a hydrophilic shell and hydrophobic core}
    }
\newglossaryentry{activation}
{
    name=activation,
    description={The process of producing porosity in a material, typically \textit{via} \gls{pyrolysis} and frequently using an \gls{activating agent}}
    }
\newglossaryentry{pyrolysis}
{
    name=pyrolysis,
    description={Thermal decomposition of material in an inert atmosphere}
    } 
\newglossaryentry{turbostratic carbon}
{
    name=turbostratic carbon,
    description={Carbons having a partially disordered structure with some ordered, graphitic domains}}
\newglossaryentry{micropore}
{
    name=micropore,
    description={Pore of width $\rm <20\ \unit{\angstrom}$}}
\newglossaryentry{mesopore}
{
    name=mesopore,
    description={Pore of width $\rm 20-500\ \unit{\angstrom}$}
}
\newglossaryentry{macropore}
{
    name=macropore,
    description={Pore of width $\rm >500\ \unit{\angstrom}$}
}
\newglossaryentry{adsorption}{
    name=adsorption,
    description={The process by which molecules adhere onto a solid surface through chemical or physical bonds}
}
\newglossaryentry{adsorbent}{
    name=adsorbent,
    description={The material onto which a substance is adsorbed}
}
\newglossaryentry{adsorbate}{
    name=adsorbate,
    description={The substance being adsorbed onto a surface}
}
\newglossaryentry{ultramicropore}{
    name=ultramicropore,
    description={A \gls{micropore} of width $\rm <7\ \unit{\angstrom}$}
}
\newglossaryentry{supermicropore}{
    name=supermicropore,
    description={A \gls{micropore} of width $\rm 7-20\ \unit{\angstrom}$}
}
\newglossaryentry{ash content}{
    name=ash content,
    description={Amount of ash, i.e. non-combustible oxides left following thermal decomposition of a sample in air. Used to determine to what degree a sample is carbonaceous, typically performed using \gls{tga}}
}
\newglossaryentry{biochar}{
    name=biochar,
    description={The product of \gls{pyrolysis}, generally produced in the absence of added \gls{porogen}}
}
\newglossaryentry{activated carbon}{
    name=activated carbon,
    description={\Gls{turbostratic carbon} that has porosity produced during \gls{pyrolysis}, aided by some external \gls{porogen}. This contrasts with \gls{biochar}}
}
\newglossaryentry{self-activation}{
    name=self-activation,
    description={\Gls{activation} that occurs due to some inherent properties of the precursor. This could be a result of oxidising agents (e.g. alkali metals) present in the precursor, or volatiles (\ce{CO2} etc.) released upon pyrolysis}
}
\newglossaryentry{physisorption}{
    name=physisorption,
    description={\Gls{adsorption} that occurs due to Van der Waals forces between \gls{adsorbent} surface and \gls{adsorbate}, as opposed to due to chemical bonding}
}