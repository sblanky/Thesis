\chapter{Conclusions \& Outlook}
\label{ch:conclusion}
The work detailed in this thesis investigated the synthesis of \glspl{turbostratic carbon} to be utilised for \ce{CO2} capture, firstly developing the understanding of carbons derived from \acrfullpl{ucf} (see chapter \ref{ch:cbs}) based on the authors previous work detailed in \ref{pub:CA} and \ref{pub:CB}. It was found that the removal of the wrapping paper on the \acrfull{ucf} is likely an essential step in the production of carbons with the extremely high porosities \textit{via} activation using \ce{KOH}, and associated gravimetric \ce{H2} capacities seen in \ref{pub:CB}. Furthermore, retaining the wrapping paper in the precursor resulted in high levels of irremovable inorganic material left in the derived \glspl{turbostratic carbon}. The resultant hierarchically porous, medium surface area carbons were found to have reasonable \ce{CO2} uptake capacity nonetheless and such materials may have further application in \acrfull{psa} applications. In addition, a series of carbons activated without the use of any external \gls{porogen} were produced, and these materials \textit{appear} to be highly ultramicroporous, though the analytic techniques used in the chapter are insufficient to yield precise and accurate data concerning porosity within this pore width range. 

Inspired by assertions that impregnated contaminant \glspl{porogen} in \acrshortpl{ucf} may provide a degree of porosity that is higher than expected from \ce{KOH} activation alone, in chapter \ref{ch:impregnation} this so-called impregnation technique was further explored. This was done \textit{via} the hydrothermal impregnation of \ce{KOH} into \acrfull{sd} prior to pyrolytic \gls{activation}, as well as by the pyrolysis of \acrfull{nc}. The relationship between porosity of the derived samples and their activation conditions was of great interest. In particular, highly ultramicroporous carbons can be synthesised \textit{via} both methods explored in this chapter. Such porosity previously been shown to be useful for low-pressure \ce{CO2} capture. However what is more interesting is the relationship between porosity of- and the synthetic conditions used- to form these novel carbons. In particular, while in general carbons derived from \acrshort{sd} are essentially entirely microporous, at a sufficiently high \ce{KOH}:\acrshort{sd} ratio the derived material becomes almost entirely mesoporous. This change in \acrshort{psd} is also accompanied by an large reduction in density, resulting in an unprecedentedly diffuse \gls{turbostratic carbon}. The \acrshort{nc}-derived samples showed unusual trends in the relationship between \gls{porogen}:precursor ratio and porosity, in that maximisation of porosity appears to occur for an \ce{Na}:\ce{C} atomic ratio of 1.2 (corresponding to a \acrfull{ds} of the sodium carboxymethyl group of 0.9). As the precursor is polymeric with carboxyl sidechains, it can be expected that cross-links may form at some point during synthesis thus yielding porosity independently from the oxidative action of \ce{Na} and \ce{Na}-containing compounds. The aforementioned breakdown in porosity indicates that these two pore forming processes may be competitive; at higher \ce{Na}:\ce{C} ratios oxidative chemical activation destroys porosity previously produced \textit{via} cross-link formation.

The synthesis-based chapters \ref{ch:cbs} and \ref{ch:impregnation} give way to further routes of investigation with respect to routes to microporous carbons for small molecule adsorption. In particular, the reason for the difficulty in removal of inorganic contaminants in \gls{ucf}-derived carbons ought to be further investigated, as the extensive washing steps used are common and have been found to be overwhelmingly successful in the community of researchers working on \glspl{activated carbon}. Indeed, the metals identified ought to be very water soluble. Routes to understanding the stubbornness of these species include use of other solvents to remove them, as well as in-depth electron microscopic techniques to understand if and whether the metal clusters are `stuck' inside the pores. Additionally, understanding of the mechanisms of porogenesis in the materials described in chapter \ref{ch:impregnation} ought to be fully elucidated, perhaps \textit{via} thermal kinetic studies, in situ electron microscopy, and/or analysis of volatiles released during the pyrolytic processes. 

In terms of analytical methods what is clear from both of the synthetic chapters is that the traditional method of porosimetry as derived from \ce{N2} isotherms measured at \qty{-196}{\degreeCelsius} is insufficient, in that \ce{N2} does not appear to easily diffuse into the \glspl{ultramicropore} present in many of the materials previously described. As such, chapter \ref{ch:dual_isotherm} details the investigation of alternative porosimetric probes, namely \ce{H2} and \ce{O2}. While these probe molecules have been investigated prior to this work, this chapter showed that only the simultaneous fit of the 2D-NLDFT heterogeneous surface (2D-NLDFT-HS) kernel to both isotherms was able to give a precise and reasonable description of the subtle \acrshort{psd} broadening as associated with increased \gls{porogen} concentration. This may be related to the fact that both \ce{O2} and \ce{H2} seem to have less trouble diffusing into these extremely small pores. Apart from this, \ce{H2} has been confirmed to probe porosity that, as a result of restrictive pore openings is not accessible to the larger \ce{O2} and \ce{N2} molecules. As the low pressure \gls{adsorption} of small, environmentally-relevant molecules such as \ce{CO2} is supposed to be associated with the presence of \glspl{ultramicropore}, the accurate understanding of porosity within such pores is vital and as such the work in chapter \ref{ch:dual_isotherm} ought to inform how porosity is measured in \glspl{turbostratic carbon}. In future these techniques should be exploited on a series of porous crystaline materials with varying \acrshortpl{psd}, but with significant porosity which is poorly accessible to \ce{N2} at \qty{-196}{\degreeCelsius}. Crystallographic data can then be compared to porosities determined as a result of \acrshort{nldft} kernel fitting to each of these isotherms and pairs thereof. This will give an indication of the accuracy of each of the isothermal porosimetric techniques which is not possible to achieve on the turbostratic materials studied in this work.

Chapter \ref{ch:pyPUC} seeks to thoroughly investigate the association of pore width with \ce{CO2} uptake as a function of pressure. In order to meticulously investigate this relationship, a small piece of software known as the \acrfull{pypuc} was produced. Starting with an experimental dataset of gravimetric \ce{CO2} uptake isotherms and \acrshortpl{psd} from a set of \glspl{turbostratic carbon}, pyPUC performs linear regressions between porosity of pores with some range of widths, and \ce{CO2} uptake at a given pressure. This process is repeated for all pore width ranges and all \ce{CO2} uptakes. As a result, a statistically optimal pore width range can be determined for uptake of \ce{CO2} at a given pressure. It was confirmed that the optimum pore width range broadened with increasing pressure, but this seems to be more associated with an increase in the upper limit, that is at higher pressures \ce{CO2} uptake becomes associated with larger and larger pores. However, past some pressure \glspl{ultramicropore} become insignificant to \ce{CO2} uptake. As mentioned in the previous paragraph, \ce{N2} porosimetry is inadequate for probing the smallest of \glspl{ultramicropore}. \acrshort{pypuc} was also able to determine that the understanding of the relationship between low-pressure \ce{CO2} uptake and ultramicroporosity is best described using dual isotherm \ce{O2}/\ce{H2} porosimetry. That is, the $r^2$ values for correlations between uptake of \ce{CO2} at $\sim$1 bar or less and the optimum pore size range are best using this porosimetric method. \acrshort{pypuc} has much more potential, in particular for the investigation of the poorly understood relationship between \ce{CH4} adsorption and pore size. In addition, the software can easily be adapted to incorporate other variables such as surface chemistry and heat of \gls{adsorption} into the understandings of the relationships determined.

In summary, this thesis presents multiple novel methods for the production of highly ultramicroporous \glspl{turbostratic carbon}, and investigates improvements to the porosimetric methodology used in their characterisation. Furthermore, a computational tool (\acrshort{pypuc}) has been created, which has helped in the thorough elucidation of the relationship between \ce{CO2} uptake and pore size. This work raises interesting questions as to the nature of pore formation mechanisms in these materials, while \acrshort{pypuc} should provide a means and philosophy by which to investigate the adsorption capacity-porosity relationship in porous materials in general.