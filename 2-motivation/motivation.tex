\chapter{Motivations \& Objectives}
\label{ch:motivations_objectives} 

As discussed in chapter \ref{ch:introduction}, \glspl{turbostratic carbon} can be useful for storing or capturing small molecules in their pores. In particular, these materials have been extensively explored for their potential for use in \ce{CO2} capture at pressures typically ranging from \qtyrange{0}{40}{\bar}, with the view to apply them to different \ce{CO2} capture regimes. It is apparent that the carbons' porosity and in particular the width of their pores plays a large role in the ability of a material to capture ambient temperature \ce{CO2}; that is at low pressures smaller pores are more useful but larger pores appear to be more beneficial at higher pressures. Furthermore, there is evidence to suggest that at lower pressures the precise value of the pore width becomes more important whereas at high pressures there is less precision needed. However the understanding of these phenomena is incomplete for a number of reasons. Firstly precisely creating \glspl{turbostratic carbon} with very narrow pores, whose widths also lie in a very narrow range is difficult, not least because the mechanisms of pore formation are a subject of debate. Secondly measurement of the widths and volumes of narrow, poorly ordered pores using current methods is complicated and furthermore appears to be quite inaccurate by current methods. Finally, the relationship between the width of the pores and the pressure-dependent \ce{CO2} uptake has only been tested at a handful of pore widths and pressures.  

To this end, this thesis investigates two relationships (i) that between synthesis conditions and porosity (especially pore size), and (ii) that between pore size and pressure-dependent, ambient temperature \ce{CO2 uptake}. In addition, improvements to current methods of determining pore width are evaluated, as precise measurement of pore width underpins relationships (i) and (ii). This investigation is to be performed \textit{via} the three routes described below.

Firstly this work exploits synthesis routes with the aim of creating carbons whose pore sizes are easily controllable in chapters \ref{ch:cbs} and \ref{ch:impregnation}. Specifically, chapter \ref{ch:cbs} aims to further the work of the author reported in \ref{pub:CB} and \ref{pub:CA} to ascertain if contaminants found in \acrshortpl{ucb} can be useful in creating porosity in carbons made from \acrshortpl{ucb} as well trying to simplify the synthesis from \ref{pub:CB}. On the other hand the work reported in chapter \ref{ch:impregnation} attempts gain a more precise understanding of the relationship between these conditions and the porosity of the products by utilising some novel synthesis methods to strictly control activation conditions. 

Secondly the objective of chapter \ref{ch:dual_isotherm} is to verify the utility of some recently developed techniques for determining porosity of carbons. Namely comparisons are made of \acrshortpl{psd} derived from fitting of the 2D-NLDFT (heterogeneous surface) kernel to \ce{N2}, \ce{O2}, and \ce{H2} isotherms as well as combinations of these isotherms. Any improvement in the accuracy of calculated \acrshortpl{psd} can then be utilised to better understand the pore formation mechanisms which take place in the novel synthesis routes discussed in chapter \ref{ch:impregnation}. 

Finally in chapter \ref{ch:pyPUC} a more thorough investigation of the relationship between pressure-dependent \ce{CO2} uptake and pore width will be introduced \textit{via} the creation of a piece of software to test the correlation between pore width and \ce{CO2} uptake at a given pressure, for every pressure in a given range. The novelty of the approach will not only be in its thoroughness, but also its use of a experimental \acrshortpl{psd} and \ce{CO2} uptake isotherms porous carbons with a high variation in their porosities. Furthermore, following on from chapter \ref{ch:dual_isotherm} more recently developed methods of determining \acrshortpl{psd} and thus pore widths will be used in order to evaluate their utility for defining the relationship between pore width and ambient temperature, pressure-dependent \ce{CO2} uptake.

More broadly this work aims to evaluate, develop, and improve tools for determining porosity in porous materials of all types, as well as for rigorous understanding of the relationship between porosity and uptake of a given \gls{adsorbent}. It is hoped that the tools developed herein will help to advance the science of adsorption in general.