\chapter{Turbostratic carbons I: from cigarette butts}
\label{ch:cbs}

\newpage

\section{Introduction}

Used cigarette butts (UCBs) pose a large environmntal hazard as a result of (i) being made of non-biodegradable cellulose acetate (CA) as well as (ii) containing a myriad of toxic chemicals.\citep{Slaughter2011, Puls2011, chevalier2018nano} As they are the most common waste material worldwide, there have been attempts to reduce their environmental presence, intially via anti-littering campaigns.\citep{Prevention2011, Harris2011} More promising however is the prospect of valorising UCBs by various means, including conversion to activated carbons as reported by the author in Publication VI as well as many other researchers.\citep{Soltani, Soltani2013, lima2018, xiong2019nitrogen, Lee2014, Hamzah2017, Yu2018, Wang2016a, Koochaki2019, Bilge2019}

The reported porosity of carbons derived from UCBs is highly varied depending on synthetic conditions. In the absence of an activating agent, and without pre-carbonisation steps, $A_{BET}$ typically does not exceed 700 $\rm m^2\ g^{-1}$.\citep{Koochaki2019, Soltani2013, Yazdi2012, Lee2014, Hamzah2017} Whereas using a porogen, and/or pre-carbonising in air or hydothermally can improve surface areas to around 3000 $\rm m^2\ g^{-1}$.\citep{xiong2018, Koochaki2019, Sun2017, Bilge2019} The author's reports of ultrahigh porosity from KOH-activated hydrothermally carbonised UCBs in Publication VI ($A_{BET}\ \rm >\ 4000\ m^2\ g^{-1} $, pore volume $\rm > 2.00\ cm^3\ g^{-1}$) and microporosity ($\rm > 90\ \%$) are partially corroborated by the similar results using pure cellulose acetate (CA) in Publication V, suggesting that CA is an ideal candidate for activation to high surface area carbons. Xiong et al also report a nitrogen-doped UCB-derived activated carbon with hierarchical porosity and $A_{BET}$ of 3420 $\rm m^2\ g^{-1}$, though this result is dubious as (i) the \ce{N2} isotherm does not include ultralow pressure data, (ii) freespace appears to be incorrectly measured and (iii) isotherm measurement is not described in the text.\citep{xiong2019nitrogen} It was suggested in Publication VI that the unusually high porosity may be contributed to by the action of so-called contaminant-porogens, i.e. trace elements in found in cigarette butts that can act as activating agents because carbons from UCBs had greater porosity than unused, i.e. 'fresh' CBs. However, another factor may be the specific treatment of the UCBs prior to any carbonisation - that is the removal of any paper, residual ash and tobacco.

The trace element composition of UCBs have been studied by various means, including neutron activation analysis of the intact butts,\citep{iskander1992multielement, Iskander1985, jenkins1985neutron, Wu1997} adsorption and emission spectroscopy of various aqueous extracts,\citep{MussaloRauhamaa1986, Kazi2009, Moriwaki2009, Moerman2011, Pelit2013, Dobaradaran2018} voltammetry experiments,\citep{Nitsch1991, Kalcher1993} as well as mixed methods according to environmental contaminant quantification standards.\citep{cardoso2018exposure} The reported concentration has a great degree of variability depending on collection site, method, and brand. For example, work by Iskander et al indicates that \ce{Al} can be present in concentrations as low as 59, and as high as 2200 $\rm{\mu g\ g^{-1}}$, depending on the country of origin of the smoked cigarette butt. Similarly, UCB samples collected from the environment\citep{Dobaradaran2017, Moriwaki2009, Moerman2011, chevalier2018nano} may have lower quantities of some elements due to leaching, but simultaneously may absorb some elements from the environment (for example from sea water). Trace elements have been identified in UCBs from almost every region of the periodic table, including alkali and alkaline earth metals;\cite{MussaloRauhamaa1986, Iskander1985, iskander1992multielement, jenkins1985neutron, Wu1997, cardoso2018exposure}  transition metals, post transition metals and metalloids;\citep{MussaloRauhamaa1986, Dobaradaran2017, Iskander1985, jenkins1985neutron, Wu1997, Moriwaki2009, Moerman2011, Pelit2013, Dobaradaran2018, Ren2017, cardoso2018exposure, chevalier2018nano} lanthanides;\citep{iskander1992multielement} and halogens.\citep{Iskander1985, iskander1992multielement, jenkins1985neutron, Wu1997} Cigarette butt derived carbons have been also been found to contain various metals in trace quantities,\citep{Soltani, Soltani2013, Yazdi2012} although \ce{Ti}, \ce{K}, and \ce{Na} have also been reported at quantities above 1 wt.\%.\citep{Soltani, Soltani2013, Yazdi2012, Lima2018, Lee2014} In addition the presence of \ce{Ca}, \ce{K}, \ce{Mg}, \ce{Na},and \ce{Al} was identified in UCB-derived hydrochar reported in Publication VI.

\section{Precursor selection, synthesis \& sample designation}
% may need to include more synthetic details depending on methodology section.

\begin{table}[b]
    \caption{Synthetic details of samples derived from cigarette butts.}
    \label{tb:cb_synthesis}
    \begin{tabularx}{\textwidth}{lXll}
        \toprule
            \textbf{Prefix} & \textbf{Preparation} & $\mathbf{T\ /\ ^{\circ}C}$ & \textbf{\# Samples} \\ 
        \midrule
            \textbf{hC}     & From public ash tray; Ash, excess tobacco removed before grinding. Hydrochar washed with $\rm 0.5\ L$ water.              & 600, 700, 800 & 9              \\
            \textbf{hD}     &  From public ash tray; Ash, excess tobacco removed before grinding.             & 600, 700, 800 & 9             \\
            \textbf{hE}     & Single brand from single smoker. Paper, ash, excess tobacco removed before grinding              & 600, 700, 800 & 3              \\
        \bottomrule
    \end{tabularx}%
\end{table}

The genesis of this synthetic work is linked to results reported in Publication VI, namely the unusually high surface area and microporosity ($ \rm > 4000\ m^2\ g^{-1}$ and $\rm > 90\ \%$ respectively) found for samples derived from the KOH-activation of used cigarette filter derived hydrochar. It was suggested in this report that this porosity was not solely a result of KOH but indeed that other porogens (so-called contaminant-porogens) may be present in the precursor. This made it important to (i) determine the identity of these  contaminant-porogens; (ii) test whether these high levels of (micro-) porosity could be attained using the whole cigarette butt; (iii) ascertain if the removal of these contaminant-porogens had an effect on porosity; and (iv) see if such contaminant-porogens do in fact confer porosity on their own. As a result, samples were synthesised both from public ash trays and from a single brand as well as with and without the wrapping paper. Full synthetic details can be found in table \ref{tb:cb_synthesis}. In all cases, used $\rm 2.5\ g$ cigarette butts (UCBs) were ground in a spice grinder, hydrothermally carbonised with $\rm 25\ cm^3$ water at $250\ ^{\circ}C$ ($\rm 5\ ^{\circ}C\ min^{-1}$), held $\rm 2\ h$ then activated for with or without KOH. After cooling, all samples were washed with HCl for at least $\rm 24\ h$, then filtered and washed with water to give neutral washings. Sample designation is h\textit{A-xTTT}, where \textit{A} is the sample prefix (see table \ref{tb:cb_synthesis}), \textit{x} is KOH:cigarette ratio (wt./wt.), and \textit{TTT} is activation temperature in $^{\circ}C$. A portion of each self activated (i.e. $x = 0$) sample in sets $C$ and $D$ was left unwashed to determine the efficacy of the final washing step. These are indicated with $'$, for example hC-0800$'$ indicates a sample activated at $\rm 800\ ^{\circ}C$ in the absence of activating agent, which was not washed. To refer to the hydrochar itself, the designation is hA-hydrochar, e.g. hydrochar derived from subset h\textit{D} (see table \ref{tb:cb_synthesis}) is hD-hydrochar.

\section{Results \& Discussion}
\label{s:cb_results}

\begin{table}[ht]
    \caption{Average yields (wt.\%) of carbons derived from the three sets of used cigarette butt samples. The yield is taken as that of the single step in the synthesis; numbers in brackets are overall yield. Where cell is blank, this sample does not exist.}
    \label{tb:cb_yield}
    \begin{tabularx}{\textwidth}{llXlXlX}
        \toprule
            \textbf{xTTT} & \multicolumn{6}{c}{\textbf{Prefix}} \\
            & \multicolumn{2}{l}{\textbf{hC}} & \multicolumn{2}{l}{\textbf{hD}} & \multicolumn{2}{l}{\textbf{hE}} \\ 
        \midrule
            \textbf{hydrochar}  & 35 & & 50 & & 39 & \\
        \\
            \textbf{0600$'$} & 38 & (13) & 39 & (19) & & \\
            \textbf{0700$'$} & 35 & (12) & 33 & (17) & & \\
            \textbf{0800$'$} & 33 & (12) & 37 & (19) & & \\
        \\
            \textbf{0600} & 34 & (12) & 36 & (18) & 6 & (14) \\
            \textbf{0700} & 30 & (11) & 31 & (16) & 30 & (11)\\
            \textbf{0800} & 31 & (11) & 33 & (17) & 24 & (9)\\
        \\
            \textbf{4600} & 8 & (3) & 13 & (7) & & \\
            \textbf{4700} & 10 & (4) & 11 & (6) & & \\
            \textbf{4800} & 9 & (3) & 10 & (5) & & \\
        \bottomrule
    \end{tabularx}%
\end{table}

% need comparison to other hydrochars and carbons
Yields of hydrochars and derived turbostratic carbons can be found in table \ref{tb:cb_yield} (refer to table \ref{tb:cb_synthesis} for information on the three sets of carbons). The fact that the yield of hC-hydrochar is significantly less than that of hD-hydrochar indicates that washing the hydrochar does remove some labile matter. Although hE-hydrochar was also not washed, it has a similar yield to hC-hydrochar perhaps indicating that the concentration of non-carbonisable material was lower in set E of cigarette butts. These differences in yield are also transferred to the overall yield (bracketed numbers) of derived turbostratic carbons, but not to the yield of the single step, indicating that whatever is removed by washing of the hydrochar does not have a significant affect on the product of the pyrolysis step. It is noteworthy that the yields of washed and unwashed carbons (0600 and 0600$'$, respectively) are very similar, indicating that the extensive washing step does not remove significant amounts of material. Yields of \ce{KOH}-activated samples are of course significantly lower than samples pyrolysed in the absence of external activating agent, due to removal of \ce{K2CO3} from the carbon framework during washing.


\subsection{Sample composition}

A primary goal of synthesising the carbons from cigarette butts was quantification and identification of so-called contaminant-porogens in cigarette butts, and monitoring their presence upon conversion of cigarette butts to hydrochar then to turbostratic carbon. Initial identification and rough quantification of the contaminants was performed using P-XRD and TGA respectively, with reference to CHN elemental microanalysis. Attempts were made to identify and more precisely quantify components using XPS and ICP-OES. Finally, imaging of the dispersion of contaminant-metals within unwashed turbostratic carbons was performed using BSE-SEM and EDX-TEM.

\begin{table}[t!]
    \caption{\ce{C}, \ce{H}, and \ce{N} content of hydrochars and carbons derived from cigarette butts, determined using elemental microanalysis as well as ash content according to residual mass following TGA in air.}
    \label{tb:chn_ash}
    \begin{tabularx}{\textwidth}{lXXXX|X}
    \toprule
        & \multicolumn{4}{c}{\textbf{Contents / wt.\%}} \\
        \textbf{Sample} & \textbf{C} & \textbf{H} & \textbf{N} & \textbf{other} & \textbf{Ash} \\
    \midrule
        \textbf{hC-hydrochar} & 56 & 5 & 0 & 44 & 8 \\
        \textbf{hD-hydrochar} & 53 & 5 & 0 & 42 & 7 \\
        \textbf{hE-hydrochar} & 61 & 4 & 2 & 32 & \\
        &&&&&\\
        \textbf{hC-0600$'$} & 69 & 2 & 0 & 29 & 16 \\
        \textbf{hD-0600$'$} & 67 & 2 & 0 & 30 & 20 \\
        &&&&&\\
        \textbf{hC-0700$'$} & 71 & 1 & 1 & 27 & 19 \\
        \textbf{hD-0700$'$} & 72 & 0 & 0 & 27 & 13\\
        &&&&&\\
        \textbf{hC-0800$'$} & 77 & 0 & 0 & 22 & 18 \\
        \textbf{hD-0800$'$} & 77 & 0 & 0 & 21 & 12 \\
        &&&&&\\
        \textbf{hC-0600} & 68 & 2 & 0 & 30 & 20 \\
        \textbf{hD-0600} & 67 & 2 & 1 & 30 & 17 \\
        \textbf{hE-0600} & 77 & 2 & 2 & 19 & 2 \\
        &&&&&\\
        \textbf{hC-0700} & 71 & 1 & 1 & 27 & 14 \\
        \textbf{hD-0700} & 72 & 0 & 0 & 24 & 16 \\
        \textbf{hE-0700} & 81 & 2 & 2 & 16 & \\
        &&&&&\\
        \textbf{hC-0800} & 75 & 1 & 1 & 22 & 13 \\
        \textbf{hD-0800} & 77 & 0 & 0 & 22 & 20 \\
        \textbf{hE-0800} & 83 & 1 & 3 & 14 & 2 \\
        &&&&&\\
        \textbf{hC-4600} & 73 & 0 & 0 & 25 & 6 \\
        \textbf{hD-4600} & 58 & 0 & 0 & 41 & 31 \\
        &&&&&\\
        \textbf{hC-4700} & 78 & 0 & 0 & 22 & 5 \\
        \textbf{hD-4700} & 53 & 0 & 0 & 46 & 5 \\
        &&&&&\\
        \textbf{hC-4800} & 90 & 0 & 0 & 10 & 6 \\
        \textbf{hD-4800} & 50 & 0 & 0 & 50 & 37 \\
    \bottomrule
    \end{tabularx}
\end{table}
% find missing values
% Maybe repeat some TGA?

Relative proportions (by weight) of \ce{C}, \ce{H}, and \ce{N} for all UCB-derived samples, as well as their ash content can be found in table \ref{tb:chn_ash}. The composition of hydrochars and carbons activated without \ce{KOH}, is essentially the same for samples in sets hC and hD. Furthermore the discrepancy in the ash content of hC-hydrochar and hD-hydrochar is within margin for error. As such, it can be confirmed that the post-hydrothermal carbonisation washing step only serves to remove combustible matter, i.e. non-metals. The slightly larger discrepancies in ash content between samples hC-0\textit{TTT} and hD-0\textit{TTT} (or hC-0\textit{TTT}$'$ and hD-0\textit{TTT}$'$) must therefore be ascribed to heterogeneous distribution of contaminants in the precursor, as opposed to removal of non-combustible contaminants prior to pyrolysis. Furthermore, washing of the turbostratic carbons does not seem to serve any consistent, significant role  in removing these contaminants. On the other hand, \ce{KOH}-activated samples (i.e. hC-4\textit{TTT} and hD-4\textit{TTT}) show more significant compositional differences, particularly in terms of \ce{C} content. This further confirms that the reduction in yield of hD-hydrochar relative to hC-hydrochar (see table \ref{tb:cb_yield}) is a result of removal of water-souble organic material not incorporated into the hydrochar macrostructure; it appears that the \ce{KOH} destroys such material in hD-4\textit{TTT}, thus reducing \ce{C} content. The higher C and lower ash content of hE-hydrochar and hE-0\textit{TTT} samples is an indication that the majority of the non-combustible contaminants seen in the ash of hC and hD samples come from the UCB wrapping paper as opposed to the UCB itself. In fact, the ash content is within margin for error for these samples, and as a result the composition of these samples is much easier to discern; the 'other' column represents the \ce{O} content. Unsurprisingly, \ce{C} content increases, and \ce{O} content decreases with increasing activation temperature as consistently reported elsewhere.\citep{Blankenship2022Modulating, Sevilla2014Energy}

\begin{table}[t!]
    \caption{Gravimetric concetrations of metals in UCB (including paper), wrapping paper and its derived, unwashed hydrochar according to ICP-OES. Samples derived from same batch of UCBs as used to make hC and hD samples - see table \ref{tb:cb_synthesis}}
    \label{tb:icp}
    \begin{tabularx}{\textwidth}{lXXXX}
    \toprule
         \multicolumn{2}{l}{\textbf{Analyte}\footnote{Numbers in brackets are wavelength measured, in nm.}} & \multicolumn{3}{c}{\textbf{Concentration / $\mathbf{mg\ g^{-1}}$}}\\
         & & \textbf{UCB} & \textbf{UCB paper} & \textbf{hydrochar}\\
    \midrule
        \textbf{\ce{Al}} & (396.153) & 1.58 & 46.0 & 4.81  \\
        \textbf{\ce{Fe}} & (283.204) & 2.81 & 38.6 & 3.66  \\
        \textbf{\ce{K}} & (766.490) & 2.91 & 17.9 & 4.60 \\
        \textbf{\ce{Mg}} & (285.210) & 0.71 & 11.0 & 1.13 \\
        \textbf{\ce{Na}} & (589.590) & 0.27 & 6.25 & 0.73 \\
        \textbf{\ce{Ti}} & (334.940) & 1.07 & 8.75 & 0.87 \\
        \textbf{\ce{Ca}} & (317.933) & 13.1 & 283 & 17.2 \\
        \textbf{\ce{Zn}} & (213.857) & - & 0.50 & - \\
    \bottomrule
    \end{tabularx}
\end{table}

%remember to make and refer to appendices
P-XRD confirms the presence of crystalline material in most of the hydrochar and turbostratic carbon samples (see appendix). Due to the complexity of the spectra it is difficult to assign peaks to any particular specie. Using XPS it was possible to identify \ce{Ti}, \ce{Na}, \ce{K}, and \ce{Ca} in dry-ashed UCBs (see appendix). although absolute concentrations are not determinable by this technique as (i) there are likely adventitious \ce{C} and \ce{O} atoms on the surface of the ash, and (ii) it is unlikely that all atomic species are accounted for. ICP-OES allowed more precise determinations of metals in UCBs, their wrapping paper, and a derived hydrochar - results are shown in table \ref{tb:icp}. These results confirm the presence of metals identified in XPS, as well as identifying \ce{Al}, \ce{Fe}, and \ce{Mg} in all samples. \ce{Zn} was only found in quantifiable amounts for the UCB wrapping paper. These metals have previously been identified in UCBs by Iskander and others.\citep{chevalier2018nano, cardoso2018exposure, iskander1992multielement, jenkins1985neutron} All trace elements were found to have a higher occurence in the paper as opposed to the whole UCB. This may explain the much higher ash content of hydrochars and turbostratic carbons derived from whole UCBs as compared to unwrapped UCBs, i.e. hC/hD  samples versus those derived from hE and those reported in Publication VI respectively. 

\subsubsection{Electron Microscopy}



\begin{table}[h]
    \caption{Concentrations of \ce{C}, \ce{O}, \ce{Al} and \ce{Ti} at three different sites in hD-4700 according to EDX-TEM }
    \label{tb:cb_edx}
    \begin{tabularx}{\textwidth}{llXlXlX}
    \toprule
        \textbf{Analyte} & \multicolumn{6}{l}{\textbf{Gravimetric concentration / wt.\% (Atomic \%)}} \\
        & \multicolumn{2}{l}{\textbf{Site 1}} & \multicolumn{2}{l}{\textbf{Site 2}} & \multicolumn{2}{l}{\textbf{Site 3}} \\
    \midrule
        \textbf{\ce{C}} & 80 & (90) & 33 & (36) & 73 & (82)\\
        \textbf{\ce{O}} & 8 & (7) & 4 & (5) & 14 & (12) \\
        \textbf{\ce{Al}} & - & (-) & - & (-) & 3 & (2) \\
        \textbf{\ce{Ti}} & 10 & (3) & 60 & (29) & - & (-) \\
    \bottomrule
    \end{tabularx}
\end{table}



\subsection{Porosity}

As a result of the discovery of high quantities of non-combustible matter in turbostratic carbons which both were and were not washed, the effect of the washing step on porosity was examined. The results, i.e. porosity of samples hD-0\textit{TTT} and hD-0\textit{TTT}$'$ are shown in table \ref{tb:cb_porosity} alongside that for sets hC-4\textit{TTT} and hD-4\textit{TTT}. It is unclear whether washing in \ce{HCl} had any effect at all on porosity - indeed, for activation at 700 and 800 $^{\circ}C$ there are significant reductions in $A_{BET}$ after washing. Perhaps this is simply a marker of physical agglomeration of particles during the washing process, and has little to do with internal porosity. Additionally, the porosity of these samples is not higher than would be expected for pyrolysed biomass (self-activation of a pure cellulose acetate-derived hydrochar yielded a carbon with $A_{BET}\ >\ 400\ \rm m^2\ g^{-1}$), so there is no proof that the non-combustible contaminants can act as porogens. Of course this is not definitive as removal of contaminants proved impossible in these samples. The porosity that hD-0\textit{TTT} carbons \textit{do} possess is principally (over 80 \%, by surface area) in the micropore region, though again this is to be expected for biochars.

\begin{table}[ht]
    \caption{Porosity of UCB-derived carbons}
    \label{tb:cb_porosity}
    \begin{tabularx}{\textwidth}{lllllll}
    \toprule
        \textbf{Sample} & \multicolumn{2}{l}{$\mathbf{A_{BET}\ /\ m^2\ g^{-1}}$}  & \multicolumn{2}{l}{\textbf{Pore volume} / $\mathbf{cm^3\ g^{-1}}$} & \multicolumn{2}{l}{\textbf{Pore size / \AA}} \\
    \midrule
        %\textbf{hC-0600$'$} & & & \\
        \textbf{hD-0600$'$} & 24 & (19) & - & (-) & \\
        %& & & \\
        %\textbf{hC-0700$'$} & & & \\
        \textbf{hD-0700$'$} & 172 & (147) & 0.08 & (0.06) \\
        %& & & \\
        %\textbf{hC-0800$'$} & & & \\
        \textbf{hD-0800$'$} & 227 & (184) & 0.10 & (0.07) \\
        & & & \\
        %\textbf{hC-0600} & & & \\
        \textbf{hD-0600} & 139 & (118) & 0.06 & (0.05) & \\
        %\textbf{hE-0600} & & & \\
        %& & & \\
        %\textbf{hC-0700} & 147 & (121) & 0.07 & (0.05) & \\
        \textbf{hD-0700} & 122 &  (101) & 0.05 & (0.04) & \\
        %\textbf{hE-0700} & & & \\
        %& & & \\
        %\textbf{hC-0800} & 94 & (75) & 0.04 & (0.03) & \\
        \textbf{hD-0800} & 102 & (84) & 0.05 & (0.03) \\
        %\textbf{hE-0800} & & & \\
        & & & \\
        \textbf{hC-4600} & 1432 & (1023) & 0.64 & (0.41) &  \\
        \textbf{hD-4600} & 1447 & (1008) & 0.62 & (0.41) & \\
        & & & \\
        \textbf{hC-4700} & 1897 & (784) & 0.94 & (0.31) & \\
        \textbf{hD-4700} & 1816 & (584) & 0.90 & (0.24) & \\
        & & & \\
        \textbf{hC-4800} & 1973 & (582) & 1.00 & (0.25) & \\
        \textbf{hD-4800} & 979 & (215) & 0.49 & (0.09) &  \\
    \bottomrule
    \end{tabularx}
\end{table}

Carbons activated using \ce{KOH} have moderate surface areas, and a much higher lower degree of microporosity, ranging from 71 \% to as low as 21 \%. That is, these carbons are much more mesoporous, and mesoporosity increases with activation temperature. Indeed, the mesoporosity is reflected in the braod curvature of the \ce{N2} isotherms used to determine these textural characteristics (see figure). This is in contrast to the carbons reported in Publication VI, where the author reported $A_{BET}$ of more than double that shown in this work, and all carbons were mostly microporous. This is likely a result of the relatively high (estimated) oxygen content, and thus low activation resistance (see Publication I, section 4.1.2.) of the hydrochars formed in this work (see table \ref{tb:chn_ash}). The lowest possible estimates of O-content for hC-hydrochar and hD-hydrochar are 37 and 35 wt.\% respectively,\footnote{Minimum O-content taken as the difference between 'other' and ash content of the hydrochars (table \ref{tb:chn_ash}). The true value is likely much higher as the ash contains the oxides of non-combustible contaminants, thus weight reported is greater than in the carbon itself.} compared to 25 wt.\% for UCB-derived hydrochar in Publication VI. 

Washing of the hydrochar does not appear to have a consistent effect on porosity of derived KOH-activated carbons. That is, for $A_{BET}$ and pore volume are the essentially the same for both sets of carbons activated with \ce{KOH} at 600 and 700 $\rm ^{\circ}C$, while porosity of hD-4800 is approximately half that of hC-4800. On the other hand, activation at 700 or 800 $\rm ^{\circ}C$ of unwashed hydrochar results in much lower absolute microporosity relative to washed hydrochar. This may be a result of combustion of volatile compounds dried into the unwashed hydrochar, thus forming oxidising gases such as \ce{CO} and \ce{CO2} which lead to uncontrolled degradation of the carbon framework, and thus pore broadening.\citep{Sevilla2014Energy, Blankenship2022Modulating} 

\subsection{\texorpdfstring{\ce{CO2} uptake}{CO2 uptake}}

While the ultra-high surface areas of carbons reported in Publication VI made them excellent candidates for \ce{H2} storage, this is not the case for UCB-derived KOH-activated carbons prepared in this work. The lower surface area and more hierarchical pore structure of these carbons (table \ref{tb:cb_porosity}) make them much better candidates for \ce{CO2} capture. As such, room temperature molar \ce{CO2} uptake was measured up to 40 bar, and results thereof are tabulated in table \ref{tb:cb_co2} and shown in full in figure xxx. In addition, to understand the effect of the contaminants in turbostratic carbons synthesised without \ce{KOH}, section \ref{sss:cb_deltaH} briefly discusses thermodynamics of adsorption of \ce{CO2} on hD-0700.

\subsubsection{\texorpdfstring{\ce{KOH}-activated samples}{KOH-activated samples}}

\begin{table}[h]
    \caption{\ce{CO2} uptakes at 1 and 20 bar for samples hC-4\textit{TTT} and hD-4\textit{TTT}.}
    \label{tb:cb_co2}
    \begin{tabularx}{\textwidth}{XXX}
    \toprule
        \textbf{Sample} & \multicolumn{2}{c}{\textbf{\ce{CO2} uptake /} $\mathbf{mmol\ g^{-1}}$} \\
         & \textbf{1 bar} & \textbf{20 bar}\\
    \midrule
        \textbf{hC-4600} & 2.6 & 9.8 \\
        \textbf{hD-4600} & 1.5 & 6.1 \\
        \\
        \textbf{hC-4700} & 2.3 & 12.1 \\
        \textbf{hD-4700} & 2.3 & 14.1 \\
        \\
        \textbf{hC-4800} & 2.7 & 12.6 \\
        \textbf{hD-4800} & 1.5 & 9.7 \\
    \bottomrule
    \end{tabularx}
\end{table}

At 20 bar, the highest surface area samples - hD-4700, hC-4700 and hC-4800 - perform the best as a result of 

% revise section
The highest surface area samples – cWB-4700, WB-4700, and WB-4800 - perform the best at high pressure where the size of the pores becomes less relevant and yields to overall availability of space for London attraction. The difference in high pressure uptake between the two samples synthesised at 700 oC is likely a result of the difference in mesoporosity. Generally, micropores are completely filled at very low pressures, and therefore their contribution to adsorption becomes obsolete as pressure increases. Thus, cWB-4700 has the better high pressure CO2 capture performance due to its slightly higher proportion of mesopores.

At lower pressure samples with lower surface area can still attain higher CO2 uptakes provided their micropore volume is reasonably high. This is evident when comparing samples WB-4600 and WB-4800. The atmospheric CO2 uptake for these two samples is very similar despite the fact that the surface are of the latter is nearly 25 \% greater. However the micropore volume of the WB-4800 is only around 60 \% of that of WB-4600. Hence for low pressure applications, it is not necessary to synthesize this material at high temperatures.

Note that for samples synthesized at 600 oC, the isotherms become linear and increases very slowly at around p = 12 bar. This is likely due to a combination of low surface area and limited development of mesopores as compared with the other samples. The remaining four samples exhibit much smoother isotherms, making them more suitable for industrial CO2 purification applications such as pressure swing adsorption.24 This is a result of having a wide range of pore sizes meaning that CO2 can be readily adsorbed at a wide range of pressures.

The CO2 uptake data for these samples somewhat mirrors the unusual behaviour of cigarette butts when activated, in that gas sorption applications are favoured by lower activation temperatures than usual.40, 94 Low pressure capture is optimised at 600 oC, whereas the material best for pressure swing adsorption is formed at 700 oC. Additionally, an activated carbon produced from unwashed hydrochar provides the best candidate of this series for PSA, meaning that the process can be further simplified via omission of the washing step.


\subsubsection{Heat of adsorption in carbon activated without \ce{KOH}}
\label{sss:cb_deltaH}

\section{Conclusions}

\bibliographystyle{rsc}
\bibliography{bibliography/bib}

\section*{Appendix}