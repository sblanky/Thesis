\subsection{Cigarette Butts}
\label{ss:cigarette_butts}

\begin{table}[ht]
\caption{Average yields (wt.\%) of carbons derived from the three sets of used cigarette butt samples. The yield is taken as that of the single step in the synthesis; numbers in brackets are overall yield. Where cell is blank, this sample does not exist.}
\label{tb:cb_yield}
\begin{tabularx}{\textwidth}{llXlXlX}
\toprule
\textbf{xTTT} & \multicolumn{6}{c}{\textbf{Prefix}} \\
& \multicolumn{2}{l}{\textbf{hC}} & \multicolumn{2}{l}{\textbf{hD}} & \multicolumn{2}{l}{\textbf{hE}} \\ 
\midrule
\textbf{hydrochar}  & 35 & & 50 & & 39 & \\
\\
\textbf{0600$'$} & 38 & (13) & 39 & (19) & & \\
\textbf{0700$'$} & 35 & (12) & 33 & (17) & & \\
\textbf{0800$'$} & 33 & (12) & 37 & (19) & & \\
\\
\textbf{0600} & 34 & (12) & 36 & (18) & 6 & (14) \\
\textbf{0700} & 30 & (11) & 31 & (16) & 30 & (11)\\
\textbf{0800} & 31 & (11) & 33 & (17) & 24 & (9)\\
\\
\textbf{4600} & 8 & (3) & 13 & (7) & & \\
\textbf{4700} & 10 & (4) & 11 & (6) & & \\
\textbf{4800} & 9 & (3) & 10 & (5) & & \\
\bottomrule
\end{tabularx}%
\end{table}

% need comparison to other hydrochars and carbons
Yields of hydrochars and derived turbostratic carbons can be found in table \ref{tb:cb_yield} (refer to table \ref{tb:cb_synthesis} for information on the three sets of carbons). The fact that the yield of hC-hydrochar is significantly less than that of hD-hydrochar indicates that washing the hydrochar does remove some labile matter. Although hE-hydrochar was also not washed, it has a similar yield to hC-hydrochar perhaps indicating that the concentration of non-carbonisable material was lower in set E of cigarette butts. These differences in yield are also transferred to the overall yield (bracketed numbers) of derived turbostratic carbons, but not to the yield of the single step, indicating that whatever is removed by washing of the hydrochar does not have a significant affect on the product of the pyrolysis step. It is noteworthy that the yields of washed and unwashed carbons (0600 and 0600$'$, respectively) are very similar, indicating that the extensive washing step does not remove significant amounts of material. Yields of \ce{KOH}-activated samples are of course significantly lower than samples pyrolysed in the absence of external activating agent, due to removal of \ce{K2CO3} from the carbon framework during washing.

\subsubsection{Sample composition}

A primary goal of synthesising the carbons from cigarette butts was quantification and identification of so-called contaminant-porogens in cigarette butts, and monitoring their presence upon conversion of cigarette butts to hydrochar then to turbostratic carbon. Initial identification and rough quantification of the contaminants was performed using P-XRD and TGA respectively, with reference to CHN elemental microanalysis. Attempts were made to identify and more precisely quantify components using XPS and ICP-OES. Finally, imaging of the dispersion of contaminant-metals within unwashed turbostratic carbons was performed using BSE-SEM and EDX-TEM.

\begin{table}[ht!]
    \caption{\ce{C}, \ce{H}, and \ce{N} content of hydrochars and carbons derived from cigarette butts, determined using elemental microanalysis as well as ash content according to residual mass following TGA in air.}
    \label{tb:chn_ash}
    \begin{tabularx}{\textwidth}{lXXXX|X}
    \toprule
        & \multicolumn{4}{c}{\textbf{Contents / wt.\%}} \\
        \textbf{Sample} & \textbf{C} & \textbf{H} & \textbf{N} & \textbf{other} & \textbf{Ash} \\
    \midrule
        \textbf{hC-hydrochar} & 56 & 5 & 0 & 44 & 8 \\
        \textbf{hD-hydrochar} & 53 & 5 & 0 & 42 & 7 \\
        \textbf{hE-hydrochar} & 61 & 4 & 2 & 32 & \\
        &&&&&\\
        \textbf{hC-0600$'$} & 69 & 2 & 0 & 29 & 16 \\
        \textbf{hD-0600$'$} & 67 & 2 & 0 & 30 & 20 \\
        &&&&&\\
        \textbf{hC-0700$'$} & 71 & 1 & 1 & 27 & 19 \\
        \textbf{hD-0700$'$} & 72 & 0 & 0 & 27 & 13\\
        &&&&&\\
        \textbf{hC-0800$'$} & 77 & 0 & 0 & 22 & 18 \\
        \textbf{hD-0800$'$} & 77 & 0 & 0 & 21 & 12 \\
        &&&&&\\
        \textbf{hC-0600} & 68 & 2 & 0 & 30 & 20 \\
        \textbf{hD-0600} & 67 & 2 & 1 & 30 & 17 \\
        \textbf{hE-0600} & 77 & 2 & 2 & 19 & 2 \\
        &&&&&\\
        \textbf{hC-0700} & 71 & 1 & 1 & 27 & 14 \\
        \textbf{hD-0700} & 72 & 0 & 0 & 24 & 16 \\
        \textbf{hE-0700} & 81 & 2 & 2 & 16 & \\
        &&&&&\\
        \textbf{hC-0800} & 75 & 1 & 1 & 22 & 13 \\
        \textbf{hD-0800} & 77 & 0 & 0 & 22 & 20 \\
        \textbf{hE-0800} & 83 & 1 & 3 & 14 & 2 \\
        &&&&&\\
        \textbf{hC-4600} & 73 & 0 & 0 & 25 & 6 \\
        \textbf{hD-4600} & 58 & 0 & 0 & 41 & 31 \\
        &&&&&\\
        \textbf{hC-4700} & 78 & 0 & 0 & 22 & 5 \\
        \textbf{hD-4700} & 53 & 0 & 0 & 46 & 5 \\
        &&&&&\\
        \textbf{hC-4800} & 90 & 0 & 0 & 10 & 6 \\
        \textbf{hD-4800} & 50 & 0 & 0 & 50 & 37 \\
    \bottomrule
    \end{tabularx}
\end{table}
% find missing values
% Maybe repeat some TGA?

Relative proportions (by weight) of \ce{C}, \ce{H}, and \ce{N} for all UCB-derived samples, as well as their ash content can be found in table \ref{tb:chn_ash}. The composition of hydrochars and carbons activated without \ce{KOH}, is essentially the same for samples in sets hC and hD. Furthermore the discrepancy in the ash content of hC-hydrochar and hD-hydrochar is within margin for error. As such, it can be confirmed that the post-hydrothermal carbonisation washing step only serves to remove combustible matter, i.e. non-metals. The slightly larger discrepancies in ash content between samples hC-0\textit{TTT} and hD-0\textit{TTT} (or hC-0\textit{TTT}$'$ and hD-0\textit{TTT}$'$) must therefore be ascribed to heterogeneous distribution of contaminants in the precursor, as opposed to removal of non-combustible contaminants prior to pyrolysis. Furthermore, washing of the turbostratic carbons does not seem to serve any consistent, significant role  in removing these contaminants. On the other hand, \ce{KOH}-activated samples (i.e. hC-4\textit{TTT} and hD-4\textit{TTT}) show more significant compositional differences, particularly in terms of \ce{C} content. This further confirms that the reduction in yield of hD-hydrochar relative to hC-hydrochar (see table \ref{tb:cb_yield}) is a result of removal of water-souble organic material not incorporated into the hydrochar macrostructure; it appears that the \ce{KOH} destroys such material in hD-4\textit{TTT}, thus reducing \ce{C} content. The higher C and lower ash content of hE-hydrochar and hE-0\textit{TTT} samples is an indication that the majority of the non-combustible contaminants seen in the ash of hC and hD samples come from the UCB wrapping paper as opposed to the UCB itself. In fact, the ash content is within margin for error for these samples, and as a result the composition of these samples is much easier to discern; the 'other' column represents the \ce{O} content. Unsurprisingly, \ce{C} content increases, and \ce{O} content decreases with increasing activation temperature as consistently reported elsewhere.\citep{Blankenship2022Modulating, Sevilla2014Energy}

\begin{table}[b!]
    \caption{Gravimetric concetrations of metals in UCB (including paper), wrapping paper and its derived, unwashed hydrochar according to ICP-OES. Samples derived from same batch of UCBs as used to make hC and hD samples - see table \ref{tb:cb_synthesis}}
    \label{tb:icp}
    \begin{tabularx}{\textwidth}{lXXXX}
    \toprule
         \multicolumn{2}{l}{\textbf{Analyte}\footnote{Numbers in brackets are wavelength measured, in nm.}} & \multicolumn{3}{c}{\textbf{Concentration / $\mathbf{mg\ g^{-1}}$}}\\
         & & \textbf{UCB} & \textbf{UCB paper} & \textbf{hydrochar}\\
    \midrule
        \textbf{\ce{Al}} & (396.153) & 1.58 & 46.0 & 4.81  \\
        \textbf{\ce{Fe}} & (283.204) & 2.81 & 38.6 & 3.66  \\
        \textbf{\ce{K}} & (766.490) & 2.91 & 17.9 & 4.60 \\
        \textbf{\ce{Mg}} & (285.210) & 0.71 & 11.0 & 1.13 \\
        \textbf{\ce{Na}} & (589.590) & 0.27 & 6.25 & 0.73 \\
        \textbf{\ce{Ti}} & (334.940) & 1.07 & 8.75 & 0.87 \\
        \textbf{\ce{Ca}} & (317.933) & 13.1 & 283 & 17.2 \\
        \textbf{\ce{Zn}} & (213.857) & - & 0.50 & - \\
    \end{tabularx}
\end{table}

%remember to make and refer to appendices
P-XRD confirms the presence of crystalline material in most of the hydrochar and turbostratic carbon samples (see appendix). Due to the complexity of the spectra it is difficult to assign peaks to any particular specie. Using XPS it was possible to identify \ce{Ti}, \ce{Na}, \ce{K}, and \ce{Ca} in dry-ashed UCBs (see appendix). although absolute concentrations are not determinable by this technique as (i) there are likely adventitious \ce{C} and \ce{O} atoms on the surface of the ash, and (ii) it is unlikely that all atomic species are accounted for. ICP-OES allowed more precise determinations of metals in UCBs, their wrapping paper, and a derived hydrochar - results are shown in table \ref{tb:icp}. These results confirm the presence of metals identified in XPS, as well as identifying \ce{Al}, \ce{Fe}, and \ce{Mg} in all samples. \ce{Zn} was only found in quantifiable amounts for the UCB wrapping paper. These metals have previously been identified in UCBs by Iskander and others.\citep{chevalier2018nano, cardoso2018exposure, iskander1992multielement, jenkins1985neutron} All trace elements were found to have a higher occurence in the paper as opposed to the whole UCB. This may explain the much higher ash content of hydrochars and turbostratic carbons derived from whole UCBs as compared to unwrapped UCBs, i.e. hC/hD  samples versus those derived from hE and those reported in Publication VI respectively. 

\subsubsection{Porosity}

\subsubsection{\ce{CO2} uptake}