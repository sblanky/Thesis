\cleardoublepage

\chapter*{Abstract} \label{Abstract}
\addcontentsline{toc}{chapter}{Abstract}

As of 2021, the atmospheric concentration of \ce{CO2} is 412 ppm and continues to rise. It is well established that these concentrations are resulting in rising temperatures, an increased incidence of extreme weather, and thus unspeakably disastrous consequences for all life. It follows then that \ce{CO2} capture and removal technologies must be rapidly developed to ensure the longevity of human society. One area of research is \ce{CO2} capture \textit{via} \gls{physisorption} onto porous materials and in particular on the easily synthesised \glspl{turbostratic carbon}. This application requires fine control over porosity (surface area, pore volume, pore size) according to conditions of sorption, and it therefore follows that both the ability to precisely measure pore sizes, as well as a definitive knowledge of the relationship between \ce{CO2} uptake capacity and pore size is needed. This thesis attempts to address all three of these issues.

In terms of routes to \glspl{activated carbon}, this work investigates two principal synthetic methods. Firstly in chapter \ref{ch:cbs} - developing on the author's previous work - a simplification of the production of \glspl{turbostratic carbon} from unwrapped \acrfullpl{ucf} was attempted by \gls{activation} of whole \acrfullpl{ucb} with \ce{KOH}. The simplified method resulted in much less porosity as compared to the previous work (maximal $A_{BET}$ of 4300 compared with 1960 $\rm m^2\ g^{-1}$) and the samples derived by this method possess a hierarchical - as opposed to narrow, microporous - \acrfull{psd}. Nevertheless, these new materials may perform well for \ce{CO2} capture in \acrfull{psa} applications. \Gls{pyrolysis} of \acrshortpl{ucb} in the absence of a \gls{porogen} created minimal porosity, perhaps as a result of contaminants present from the \acrshort{ucb} wrapping paper. 

The other approach to activation used (chapter \ref{ch:impregnation}) are methods which have been collectively coined as impregnation routes, i.e. methods which attempt to achieve close contact between precursor and the activating species whilst maintaining a homogeneous distribution of the latter throughout the former. Impregnation was achieved through (i) the hydrothermal carbonisation of \acrfull{sd} with \ce{KOH} prior to pyrolysis, as well as (ii) direct activation of a polymeric sodium salt, \acrfull{nc}. In both cases, \acrshortpl{psd} achieved were generally narrow and situated principally in the small \gls{micropore} region, making the products potential candidates for low pressure \ce{CO2} capture. Both sets of materials also showed unexpected features, \acrshort{sd}-derived samples having extremely low bulk density, and those obtained from \acrshort{nc} exhibiting reduction in porosity at surprisingly low \gls{porogen}:precursor ratios. The latter of these suggests pore formation effects outside of the caustic nature of \gls{porogenesis} with \ce{Na} compounds, and is a potential route for further investigation of \gls{activation} mechanisms.

For materials derived through the synthetic routes mentioned above, reliable, accurate, and efficient isothermal porosimetry proved difficult due to poor diffusion of \ce{N2} into the materials' pores. As such, alternative porosimetric techniques were investigated in chapter \ref{ch:dual_isotherm}. It was found that dual isotherm porosimetry using \ce{O2} and \ce{H2} isotherms at -196 $\rm ^{\circ}C$ results not only in more expedient equilibration of the sorptive-sorbent system, but allows the measurement of sub-angstrom level developments in porosity associated with changes in quantity of \gls{porogen} used. These subtle developments in porosity are not measurable through traditional porosimetry using \ce{N2}. 

As for improving the understanding of the relationship between \ce{CO2} uptake as a function of pressure and pore size, chapter \ref{ch:pyPUC} details the development and deployment of the \acrfull{pypuc} which, using experimental \acrshortpl{psd} and gravimetric gas uptake isotherms applies a brute force approach to determine the correlation between porosity within some pore width range and \ce{CO2} uptake at a given pressure. This is performed for all user-defined pore width ranges and pressures, and correlation coefficients are compared to give optimum pore size ranges, $\Omega$ at each pressure. When applied to \ce{CO2} uptake on \glspl{turbostratic carbon}, it was confirmed that $\Omega$ broadens with increasing pressure, albeit to a more granular level of detail than previously reported. Furthermore, following on from findings in chapter \ref{ch:dual_isotherm} it was found that these relationships at low pressures are best described using dual isotherm \ce{O2}/\ce{H2} porosimetry.